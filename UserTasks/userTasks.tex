\subsubsection*{Getting in touch with the users}

We want to make sure our app meets, some of the requirements of real potential users for our product. If the app doesn’t do that, they won't use it. To secure that, we intended to make a walks-through, with several potential users, to create a list of things, the users would actually want to do with the system.\\

In \ref{TFI}, we see several screen shots from the developed app. The first interface presents the log in window. After typing in the specific log in information, the user can access the homepage that are described in chapter \ref{DisplayOfAuctions}.\\

If the user wants to create an auction, he/she would have to tap, at the camera icon in the down right corner. The user can choose between taking a real picture, with the frame recognising camera, or pick by her/his photo library. After has chosen the specific photo, the user must type in some information about price and dimensions. 

\subsubsection*{Comparison}

We have tested a number of other similar applications among the study group. These tests were intended to clarify the process, that are present when creating an action or an advertisement on one of the competitors' app. We can conclude that the process that has been ongoing, is everything from long and difficult, to easy and manageable. Depending on which competitors we have observed, the process was different. We consider ours as a real winner for the purpose we have set up. 

\subsubsection*{Learning about the user tasks}

With our application, we transferred the standardised user habits. The purpose for this, was not having changes in the usual user interaction with the systems. We know, that innovative interactions between humans and systems may have shared opinions. One segment likes it and another simply can't adapt the new way to interact with it\footnote{\url{https://www.theverge.com/2017/10/31/16579748/apple-iphone-x-review}}. By our application, they don’t have to learn anything new, and it has only been easier to apply, as we have minimised the process of creating an online auction.

We have worked with the traditional way, to ensure that all important features of our system are well-known to the users. We have done that, to ensure a coherent and well-functioning app that works across all features.
