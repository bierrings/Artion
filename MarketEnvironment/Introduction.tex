\section{Market environment}
Throughout this report, we concern the market for art. In this chapter, we will elaborate the environment of the market to enter, hereby the competitors and the potential positioning of Artion. 

\subsection{Competitors}
The market consists of a wide range of competitors, but we consider respectively Lauritz.com, Bruun Rasmussen, DBA, QXL and various social media platforms to be the greatest and nearest competitors in relation to Artion. The reason why social media also constitutes competition is, among other, due to the increasing popularity in the use of sales groups on Facebook. Furthermore, Instagram has also become a popular channel for sales. 
\\\\
The stated competitors above can be distributed into three groups:

\begin{enumerate}
\item Lauritz.com and Bruun Rasmussen
\item DBA and QXL
\item Facebook and Instagram
\end{enumerate}

As illustrated in appendix \ref{PositioningMap}, group 1 represents a high level of time to market, but a fairly short time spent searching for products. The reason for the high level of time to market, is the extensive process for the seller to create an advertisement. A product must be handed to the company for valuation, picturing, storing and further handling. However, the time spent searching for products (for the potential customers) is quite low, because of the simplicity, consistent content and the time frame on auctions, ensuring there is no abundance of old products on the platforms.\\
\forceindent Group 2 represents a low level of time to market but much time spent searching for products. It is quick and easy to create an advertisement, and there are no requirements in terms of product category, quality and price. This results in an abundance of products. Another issue for this group, which only applies for DBA, is that there is no time frame for the products, contributing to an even greater abundance of old products.\\
\forceindent Group 3 represents a low- to mid level of time to market and much time spent searching for products. Using Instagram for sales purposes is very quick and easy, but this only applies with a lot of followers as a prerequisite. Facebook has a slightly higher level of time to market, as is takes time and might be difficult to find the appropriate sales groups before creating an advertisement. The time spent searching for products (for the customers) for both Instagram and Facebook is high, as it might be very time consuming to find the right groups, hashtags, sites etc.

\subsection{Positioning}
In continuation of the above mentioned aspects, we identify a gap in the market, forming an opportunity for Artion. The need we strive to cover is two-sided; the buyer and the seller. The need of the buyer is a consistent, inspirational and user-friendly platform to buy paintings. The need of the seller is a platform that allows quick creation, thus short time to market, of advertisements, which exposes the paintings for sale without getting lost in the abundance of other products. The combination of these two needs represents the foundation for Artion. The way in which the application will meet these needs, is further elaborated in chapter \ref{ValueProposition}. Currently at DBA, there are more than 15.000 advertisements in the category "Paintings"\footnote{https://www.dba.dk/til-boligen/kunst-og-antikviteter/malerier/}, and we are very certain that a large proportion of these advertisements are due to the lack of a better and more inspirational platform. Furthermore, there seems to be an increasing interest in art among the danish population, with an increase of more than 1 million visitors to the danish art museums in the years 2010-2016, illustrated in appendix \ref{Visitors}. This trend might indicate a growth in the market for art. These, among others, are reasons for our choice of positioning, illustrated in appendix \ref{PositioningMap}.