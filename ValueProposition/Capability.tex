\subsubsection{Frame recognising camera}
\label{FrameCamera}
The frame recognising camera is one of the essential features that makes our application unique. When a user creates an auction, he must use the built-in camera function that recognises the frame of the painting and only captures what is within the frame. In this way we ensure a consistent presentation of all auctions. This process is illustrated in figure \ref{CreateAuction}.

\subsubsection{Simplicity}
One of the reasons for us to develop this app, is that the existing platforms at the current market are not simple and easy to use. Lauritz.com provides a quite easy and simple application. They manage all of the auctions, including valuation, picturing, storing in showrooms and distribution of products after hammering. We think Lauritz.com provides a simple and easy to use application, but the associated processes are very difficult for both the seller and the buyer, in terms of handling and pickup. 

\subsubsection{Safe transactions}
With inspiration from services like AirBnB and Uber, we introduce the transaction-confirmation feature. All transactions and payments are managed by the application. When an auction has ended, the payment is locked in the system. The seller will be notified when the buyer has payed for the product, but the seller will not receive any money before the painting is received by the buyer. We have implemented an interface that enables the buyer and seller to confirm that the exchange of the product has taken place, whereafter the seller receives the money.  