In the course Innovation and new technology, innovation is defined as follows: “production or adoption, assimilation, and exploitation of a value-added novelty in economic and social spheres; renewal and enlargement of products, services, and markets; development of new methods of production; and establishment of new management systems. It is both a process and an outcome”.  

With this state of mind, one of the group members came up with an idea of a technological innovation on behalf of his interest in art. In the current market for trading art, many opportunities exist but various disadvantages follow each one of them. Together the group collaborated on further development of this idea and created the concept of Artion together.

Artion is an online platform for trading art. The application has two purposes. First it seeks to provide a platform making it quick and easy for private people as well as artists and art-distributors to sell paintings. Secondly, to be an inspiring platform for people who wants to buy paintings. \\

We believe the market is missing a platform combining the two aspects stated above. The current market provides following two models:
\begin{enumerate}
    \item Quick and easy in terms of selling products, but not inspiring in terms of buying products
    \item Time-consuming and difficult in terms of selling products, but inspiring in terms of buying products
\end{enumerate}
We identify an opportunity for Artion by combining the advantages of these two models. This represents the background for initiating the project.\\

In this report we will elaborate and analyse various aspects such as the market environment, potential positioning, value proposition and user tasks. Furthermore, we will clarify our considerations regarding human-computer interaction and explain how data is handled by our vertical prototype. Finally, we will present our business model and enlighten a range of commercial challenges.

