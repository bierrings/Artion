Based on the analysis of the current market for trading art, we identified a gap that potentially can generate value. The gap occurs because of the existing trading platforms lack of providing
the necessary service to cover the needs we have identified, for people interested in art. On behalf of this assumption, we utilized software development in order to develop a platform called Artion.\\ 

Artion seeks to meet the market segment by providing innovative features that ensures an easy to market gateway. In order to secure adoption of the application, we designed the interface to support the existing conceptual models from the current platforms used for selling art. In order to commercialise Artion, we conducted a viable business model to mediate between the technical and economic domains. As Artion is still in the development stage, the conducted business model takes starting point in the expected costs and profit to occur when launching the finished product. Furthermore, various strategic considerations regarding establishment of innovation was composed to an overall strategy, in order to obtain competitive advantages. In extension of this, a list of potential commercial challenges was created, for better management of the challenges Artion might meet. In conclusion of the above, we believe that Artion will generate value in a real practical scenario. 