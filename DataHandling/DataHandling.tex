In this chapter, we will briefly describe how the application-prototype handles data, including how it displays, saves and loads the data. Additionally, we will shortly enlighten how the data should be captured if the application were to go live. 

\section{Data in the prototype}
The database used for the prototype is developed with Firebase\footnote{https://firebase.google.com/}, a real-time database enabling to store and sync data in real-time across all connected clients. The vertical prototype comprises data such as user data, image files, attributes for paintings etc. In terms of authentication, we have only enabled email/password as sign-in provider. The user information is encrypted. \\

The vertical prototype enables the user to create an advertisement with 'title', 'author' and 'dimensions' as attributes, simultaneously with an image attached. The picture, including the attributes, is then stored in Firebase, whereafter it is loaded and integrated into the grid of auctions in the application. The users in the prototype are hard-coded, but the auctions are created through the application. The data structure for auctions in the database consists of one parent layer containing the attributes as child items. 

\section{Considerations on data capturing}
If the application were to go live, the data should be captured different from how it is captured in the current prototype. Firstly, we use dummy data for users, since the prototype is missing an interface for creating users. If the prototype was a complete platform, it should contain an interface enabling new users to create accounts. Furthermore, the pictures attached to the auctions in the prototype are dummy data. In a final product the images should be captured by the previous described frame-recognising camera feature.