The design of Artion is assembled by fragments of other well-known concepts and re-tailored to the extent of the expected users needs. This approach secures a consistent design for our users as it appeals to their cognitive thinking. The design is also minimalistic as we want the user to think about Artion as a gallery providing the frames for art that speaks for itself.
\subsubsection*{Context}
Various pages of Artion has to connect seamlessly. As one of Artion’s main purposes is to solve the time-consuming barriers of trading art, the process in navigating through the app should be easy to adapt and remember. Assumptions about how many levels and how many choices pr. level is necessary to ensure this. Therefore, as a general contextual design, we want to have few levels and more choices pr. level. This is illustrated in appendix \ref{HorizontalPrototype}.
\subsubsection*{Content}
It should be easy to separate content. Our focal point for value creation starts when users scroll through the grid with listed art. Since the app is distributing art it is important that each expression of a painting is able to be captured. Therefore, the paintings becomes the frames of each item in the grid, as illustrated in figure \ref{AuctionsGrid}. Another key element in the concern of content is the possibility to go further if the auction being exposed is not appealing. Therefore, the scrolling is continuous and only stops whenever the user selects an auction, or if there are no more auctions in the grid - same concept as in Instagram. We also want to ensure that the user nor is being distracted by information that is not necessary. Therefore, the user will only see information regarding price and time left for the particular auction. This leads to a potential problem because the grid is defined by art uploaded by the users. One way to control this is to make a standard frame for uploading auctions. This makes the grid more presentable as it follows one standard. This potential issue is addressed in section \ref{FrameCamera}.
\subsubsection*{Community \& communication}
To some extend Artion is a community, as it is a focal point for people with interest in art. However, as Artion is a demo there will not be a vertical prototype of the internal communication system, as this is considered as an indirect value adding activity. An example of how the community of Artion could look like, is presented by having personal profiles with galleries of art and an option to search for other users. See appendix \ref{HorizontalPrototype}.

\subsubsection*{Customisation \& connection}
As mentioned previously, we want to capture only the necessary information. There are many various categories in art and therefore a user should be able to tailor the need of information by sorting categories. In this way information is restricted to the needs of the user. However, one could argue that when Artion is launched it has a low number of users, which makes categorising unnecessary and maybe even frustrating, because of less probability in finding art matching the search criteria. Therefore, sorting for information should follow the level of information exposed. As a start, users will only be able to customise information with a pin-function, enabling the user to follow auctions.

\subsubsection*{Commerce}
Commerce is vital for the trust between buyers and sellers. Regarding the design, it is important that both buyer and seller is notified when an auction is won. See horizontal prototype in appendix \ref{HorizontalPrototype}. Furthermore, it should be visible that a transaction is locked by Artion until the buyer and seller has confirmed the hand-over. If the seller does not respond, the transaction will be cancelled and the amount will be reversed to the buyer. To secure trust between buyer, seller and Artion as a third part, it is important that information about every step in this process is accessible.