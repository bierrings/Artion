The design of Artion is assembled by fragments of other well-known concepts and re-tailored to the extent of Artion’s needs. This approach secures a consistent design for our users as it appeals to their cognitive thinking. The design is also minimalistic as we want the user to think about Artion as a gallery providing the frames for art that speaks for itself.
\subsubsection*{Context}
Different pages of Artion has to connect seamlessly. While one of Artion’s main purposes, is to solve the time-consuming barriers of trading art, the process in navigating through the app should be easy to adapt and remember. Assumptions about how many levels and how many choices pr. Level Is necessary to ensure this. Therefore, as a general contextual design, we want to have few levels and more choices pr. Level.  (see figure. ?)
\subsubsection*{Content}
It should be easy to separate content. Our focal point for value creation starts when users scrolls through the grid with listed art. Since the app is distributing art it’s important that each expression of a product is able to be captured. Therefore, the art becomes the frames of each item in the grid. Another key element in the concern of content is, that it should be easy to go further if the auction being exposed isn’t appealing, therefore the scrolling is continuous and only stops whenever the user selects an auction, or if there are no more auctions in the grid. Same concept as in Instagram. We also want to ensure that the user nor is being distracted by information that isn’t necessary, so the user will only see information about price and time left for the particular auction. This leads to a potential problem because the grid is defined by art uploaded by the users. One way to control this, is to make the user select a standard frame when creating an auction, which makes the whole grid more presentable as it follows one standard. (Various assumption about improving this standard will be described on more details later.)
\subsubsection*{Community \& communication}
To some extend Artion is a community, as it is a focal point for people, with interest in art. However, as Artion only is a demo there won’t be an internal communication system for the users. The community of Artion is expressed in having personal profiles with a gallery of art and an option of searching for other users. 
\subsubsection*{Customisation \& connection}
As mentioned earlier we want to capture only the necessary information. There are many different categories in art and therefore a user should be able to tailor the need of information by sorting for categories. In this way information is restricted to the buyer’s needs. However, one could argue that when Artion is launched it has a low number of users which makes a categorising unnecessary and maybe even frustrating (because of less probability in finding art matching the search criteria).Therefore, filtering for information should follow the level of information exposed. As a start, users will only be able to customise information with a pin-function, where you can follow auctions. (see picture).

\subsubsection*{Commerce}
Commerce is vital for the trust between buyers and sellers. Regarding the design, it is important that both buyer and seller is confirmed when an auction is won. see auction won mock-up. Furthermore, it should be visible that a transaction is locked by Artion until the buyer has confirmed the transaction. If the seller does not respond, the transaction will be cancelled and the amount will be reversed to the buyer. To secure trust between buyer, seller and Artion as a third part it’s important that information about every step in this process is accessible. (see diagram for safe transactions)

(same concept as Airbnb)\\