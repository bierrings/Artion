\subsubsection{Value proposition}
Artion is a trading platform and a community that connects interests of buyers and sellers. Upcoming artist often see breaking the barriers as the hardest part as it can be difficult to address their customers. Art-distributers looks for ways to distribute art to customers with a minimum cost to maximise profit. Together both want to access customers with a low cost. As explained in the market analysis there is various ways to promote art but there isn’t a platform that connects everything together. The value proposition of Artion is to connect everything in one application. Artion is not only for trading art, it’s community for creative artist who wants to share ideas and get inspired among profiles of other artist.

\subsubsection{Market segment}
The market consists of people having an interest in art, but with various agendas. We divide the market into three segments; artists, gallerists and private people. The proposition will be appealing in general, because of the ease of use, appealing interface and the inspiring platform. On the other hand, we acknowledge that our users could have various agendas. 
Artion reach both the amateur artist who creates art as a hobby and professionals that makes a living out of it, as Artion handles trading and provides personal profiles which can be helpful in establishment of a name. We also want to reach to the gallerists, but their agenda is usually more focused on making profit, so they might find the low cost of use as the appealing element. At last Artion is appealing to private people because of the easy to market proposition.
This means that the value generating mechanisms can vary and, therefore, Artion serves various purposes. Some might only use it for trading, some might use it to be a part of the community and some might use it for both trading and the community.


\subsubsection{Value chain structure }
Artion’s structure of value chain comes in two categories. As a trading platform and a community. We prospect these two as the reason a buyer would pay for our service.

\subsubsection{Artion as a trading platform}
Artion seek to reduce the barriers of selling art. The cost of selling art is hard to measure because it depends on many variables. For example promotional skills of the artist or art-distributer, the demand for art, popularity and so on. In this matter, we identify three teleological issues; finding an audience, expanding an audience and keeping an audience. If these are supported to an extend where the three segments effort in using the platform, is less the value generated from creating an auction, we believe that there are incentives for using Artion and value can be captured for both our users and Artion. 

\subsubsection{Artion as a community}
We believe that having a community can establish value for our segment. As we identified in our market analysis, many artists distribute art through different medias and platforms. If the trend is to promote art through these, then our trading platform should include a community to support the needs of our segment.\\

When these two connects the market becomes more visible for everyone and thus a decrease in cost of selling art. When this is based on the prospected needs of the market, we identify our segment as our assets needed in order to support our position. Therefore, users are crucial to the value structure of Artion.

\subsubsection*{Cost structure \& profit potential}

As indicated in the value proposition and the structure of value chain we see that users and value is causally connected. Artion charges 4\% in commission for every auction sold through the platform. This identifies Artion’s revenue streams as a retail revenue model, as Artion work as a third part in the transaction of an auction. Both pros and cons follow this type of revenue stream. Pros in the sense of a predictable revenue stream and cons in sense of being projected to appealing product lines, which in this situation can be specific art. 

\subsubsection{Cost expected to occur}
In commercialization of Artion is digital marketing & online advertising, events, support and use of firebase. Cost of development is not taken into account, as we expect to develop the app ourselves. Is hard the measure a budget of these as no truth from historical data from artion exists. However various expectations follow.

\subsubsection{Digital marketing and online advertising}
We expect to prioritize this in the beginning because of the crucial importance of recruiting members. Most providers like google, Facebook Instagram has different scalable solutions so information about this cost obtainable.

\subsubsection{Data storage and Firebase}
When testing our app, we saw that one hour of distributing data between our client and firebase exceeded the 1 GB trail period of firebase. Unfortunately, firebase only tracks 30 days of storage-information, so this information is no more accessible and in lack of user test, this is the only historic-user data to forecast a potential data handling progress. However, this is important in considering pricing plan from firebase. If we can exceed 1 GB data in hour of testing, something might indicate that we are vulnerable for burst loads. We identify various reason for this occurrence. One is our compression of pictures in react-native. As we use fetch-blob to collect binary data for storage, the sizing of a picture has an impact of big a stored string is, which is directly traceable as a cost in any of the pricing plans. Therefore, we expect to work on a solution that compresses data more efficient in order to lower cost.
\subsubsection{Support and events}
Cost will occur when promoting events and supporting our customers. We expect the cost of support to be a variable cost of users and the cost of events to be fixed.
\subsubsection{Vaue network}


\subsubsection{Competitive strategy}
formulate the competitive strategy by which the innovating firm will gain and hold advantage over rivals. 

 - måske noget om platform thinking og service innovation. Slide 07 og 08