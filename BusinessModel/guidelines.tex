In order to create a sustainable revenue stream from our innovation, we have developed a business model based on Henry Chesbrough and Richard S. Rosenbloom definition of requirements \cite[553]{BusinessModel}. According to Chesbrough \& Rosenbloom the general idea of a business model is phrased on a high level of abstraction and lacks on details. They claim that a business model mediate between the technical and the economic domains. In between these domains they formulate a detailed and operational definition of what a business model should present in order to link the two domains \cite[536]{BusinessModel}. An overall identification of business models in the market is illustrated in appendix \ref{BusinessModels}.

\section{Market segment}
\label{Segment}
The market consists of people having an interest in art, but with various agendas. We divide the market into three segments; artists, gallerists and private people. We have made three personas to substantiate the chosen segments, illustrated in appendix \ref{Personas}. The proposition will be appealing in general, because of the ease of use, appealing interface and the inspiring platform. On the other hand, we acknowledge that our users could have various agendas. 
Artion reach both the amateur artist who creates art as a hobby and professionals that makes a living out of it. We also want to reach to the gallerists, but their agenda is usually more focused on making profit. At last Artion is appealing to private people because of the easy to market proposition.
This means that the value generating mechanisms can vary and, therefore, Artion serves various purposes. Some might only use it for trading, some might use it to be a part of the community and some might use it for both trading and the community.

\section{Value proposition}
Recall the elements of the value proposition described in chapter \ref{ValueProposition}, as these influence users of Artion in different ways. Upcoming artists might see addressing their customers as the hardest part. Gallerists looks for ways to distribute art to customers with a minimum cost to maximise profit. Private people might not have a clue where to begin\footnote{"Importantly, different prospective customers may desire different latent attributes of the technology. Thus, there is no single inherent value for the technology: if it subsequently were to be developed in different ways, it would likely accrue different value to its developer." - Chesbrough, H., and Rosenbloom}. We define the described value proposition as the element that makes it easier for our segment to access customers with a low cost. The value proposition of Artion can then be summarised as; easy to market, inspiring to buy. 

\section{Value chain structure}
Artion’s structure of value chain comes in two categories. As a trading platform and as a community. We prospect these two categories as the reasons why people would use our service.

\subsubsection{Artion as a trading platform}
Artion seeks to reduce the barriers of selling art. The cost of selling art is hard to measure because it depends on many variables. For example promotional skills of the artist or art-distributer, popularity, the demand for art and so on. In this matter, we identify three teleological issues; finding an audience, expanding an audience and keeping an audience. If these are supported to an extend where the three segments effort in using the platform is less than the value generated from creating an auction, we believe there are incentives for using Artion. In this way, value can be captured for both ourselves and users of Artion. 

\subsubsection{Artion as a community}
\label{Community}
We believe that having a community can establish value for our segment. As we identified in our market analysis, many artists distribute art through different medias and platforms. If the trend is to promote art through these, then our trading platform should include a community and support the needs of our market segment.\\

When these two connects, the market becomes more visible for everyone and thus a decrease in cost of selling art. When this is based on the prospected needs of the market, we identify our segment as our assets needed in order to support our position. Therefore, users are crucial to the value structure of Artion.

\section{Expected profit and cost}

\subsection{Expected profit}
As indicated in the value proposition and the structure of value chain, we see that users and value is causally connected. Artion charges 4\% of the hammer price and a fee of 20 kr. in commission for every auction sold through the platform. This identifies the revenue stream of Artion as a retail revenue model, as Artion acts as a third part in the transaction of an auction. Both pros and cons follow this type of revenue model. Pros in the sense of a predictable revenue stream and cons in the sense of being projected to appealing product lines, which in our situation can be trending art.

\subsection{Expected cost}
Recall section \ref{Cost}, a commercialisation of Artion follow various costs. Cost of development is not taken into account, as we expect to develop the app ourselves. It is difficult to determine a budget for Artion, because no truth from historical data exists, as the application has not launched yet. However, various expectations follow.

\subsubsection{Digital marketing and online advertising}
We expect to prioritise this in the beginning because of the crucial importance of recruiting users. Most providers like Google, Facebook and Instagram provides various scalable solutions for digital marketing and online advertising, which makes information about cost easy to measure.

\subsubsection{Data storage and Firebase}
In the process of development, we experienced that one hour of distributing data between our client and Firebase exceeded the 1 GB period of trial. Unfortunately, Firebase only tracks 30 days of storage-information, thus this information is no more accessible. In lack of user tests, this is the only historical user data to forecast for a potential data handling progress. However, this is important in considering the pricing plan from Firebase. If we exceed 1 GB data in one hour of testing, it might indicate that we are vulnerable for burst loads. We identify various reason for this occurrence. One is our compression of pictures in react-native. As we use fetch-blob to collect binary data for storage, the size of a picture has an impact of how great a stored string is in Firebase. This is directly traceable as a cost in any of the pricing plans. Therefore, we expect to work on a solution that compresses data more efficient in order to lower cost.
\subsubsection{Support and events}
Costs will occur when promoting events and supporting our customers. We expect the cost of support to be a variable cost in terms of the amount of users and the cost of events to be fixed.


\section{Value network}
As we consider ourselves as first movers in providing a trading platform with a community, we expect that people who has weight in the art community to be important influencers and as a part of our value network. Therefore, it is important for Artion to adapt to especially these people. At the same time, we expect private people to be the vast majority and therefore it is important that our platform builds on recognisable concepts in order to reach widely.

\section{Competitive strategy}
To secure anchorage of Artion, various strategic initiatives and general assumptions about the market will be presented in the following section. The choice of strategies reflects patterns of how Artion can leverage on its core competences to secure competitive advantages in the market.
\\\\
In strategic typology, Artion can be seen as a combination of the prospector and the analyser. Artion seeks to leverage on the opportunities of the current market of art, as we want to improve the time to market by connecting the dots. This could potentially create a new and more consistent market. However, one could also argue that the idea of Artion builds on existing concepts and therefore has a more analytic and sustaining approach.
\\\\
One of the key strategies to establish value creation is by using wisdom of the crowd \cite[339]{WiseCrowd}. By creating a community Artion can leverage on the collective intelligence from artists, gallerists and private art distributors sharing their ideas. As Artion has various users with different agendas we see our users of the community as mobs who contribute as individual specialists with relatively homogeneous interests because of their general interest in getting inspired among others in the community \cite[348]{WiseCrowd}.  
\\\\
To recruit users, various strategies can be implemented. As we start from point zero, our approach is to ask for volunteers, which according to AnHai Doan \& co. is the most popular solution \cite[93]{Crowdsourcing}. Our strategy is to screen the market for potential users of the community. This process can be performed by screening different social media for groups and communities of art, but also by promoting physically at vernissages and exhibitions. Another approach is to host events where people can try the app. At the events, we will be offering free setup and guidance of how the platform works in order to establish a relationship with the users. A third way to reach the potential users is through social media and sponsored campaigns but as mentioned, costs will follow this approach and therefore it is a limited resource.\\
\forceindent In this matter the interface of our application becomes very important. As AnHai Doan \& co formulate it: "The user interface should make it easy for users to contribute. This is highly non-trivial." \cite[95]{Crowdsourcing}. Recall chapter \ref{HCI} regarding design sketches in order to understand how we expect to secure successfull HCI.
\section{Overall strategy}
Together these considerations can be formed into the following overall strategy:

Artion wants to secure the user segments by not only creating a platform to distribute their art, but also to create a community where benefits of getting inspired and sharing art can be redeemed. Value creation for Artion is based on user-activity and therefore it is important to prioritise the users first. in order to encourage users to contribute, an appealing user interface and good user experience is essential. 
